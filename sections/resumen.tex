
        En la industria tecnológica, es muy importante la ciberseguridad. Las empresas cada vez más se centran en garantizar que sus distintas aplicaciones, proyectos y servicios son seguros. Que están protegidos de amenazas internas, externas y vulnerabilidades. Incluso están empezando a adaptar sus soluciones en tecnologías que garanticen una mayor seguridad y robustez posible, como es la tecnología Blockchian.\\
        
        Por otro lado, las empresas deben proteger la seguridad de la información, garantizando la confidencialidad, disponibilidad e integridad de la información. Este problema muchas veces va ligado con normativas de protección de datos, como puede ser la \acrshort{gdpr}. Por este motivo entre otros, cada vez suena con más fuerza la idea de "Identidad Digital Soberana". La Identidad Digital Soberana (\textit{\acrlong{ssi}} en inglés) es el concepto de que los individuos u organizaciones tienen la propiedad única y excusiva de sus identidades digitales, y son ellos mismos quienes controlan la forma en que se comparten y utilizan sus datos personales.\\
        
        Alastria, un consorcio que fomenta la economía digital a través del desarrollo de tecnologías de registro descentralizado como es Blockchain, ha definido un modelo de identidad digital llamado \textit{Alastria ID}. El proyecto \textit{Alastria ID} está desplegado como una de las aplicaciones básicas de la infraestructura blockchain promovida por el consorcio dentro de su plataforma. Esta propuesta tecnológica de identidad digital en blockchain tiene como objetivo proporcionar un marco de infraestructura y desarrollo, para llevar a cabo proyectos de Identidad Digital Soberana, con plena vigencia legal en la zona euro.\\
        
        % Por otro lado, la tecnología \textit{Blockchain} es una tecnología que, basándose en cálculos criptográficos, ofrece un libro de cuentas distribuido (llamado \textit{ledger}) en el que se apuntan transacciones entre participantes que no tienen una relación de confianza entre si. Las principales características de este \textit{ledger} son que se distribuye entre los múltiples nodos participantes de la red, que es inmutable, no repudiable y que no depende de relaciones de confianza entre participantes ni de una entidad central. Por estos motivos, la mayoría de implementaciones de Identidad Digital Soberana están basadas en esta tecnología.\\
        
        Que en una empresa se utilice tecnología Blockchain, no implica que esté ausenta de problemas de seguridad y vulnerabilidades. Que un proyecto esté basado en la Identidad Digital Soberana, no implica que garantice la seguridad de la información. Al ser tecnologías nuevas, aún no hay muchos estándares, no hay implementaciones robustas, y un pequeño fallo puede ocasionar una brecha de seguridad, y el impacto que esto puede tener en un modelo de Identidad Soberana puede ser crítico para el ecosistema.\\
        
        El objetivo de este trabajo es multiple. Se ha estudiado el concepto de \textit{\acrfull{ssi}}, así como la implementación de Alastria llamada \textit{Alastria ID} y otras implementaciones y soluciones que existen en la actualidad. Por otro lado, se ha realizado un estudio desde el punto de vista de la seguridad, de la implementación de \textit{Alastria ID}, auditando los Smart Contracts diseñados en el \acrshort{mvp}1 y la librería que facilita su uso. Por último, se ha creado una prueba de concepto de una de las vulnerabilidades encontradas durante la fase de análisis y se ha evaluado su criticidad e impacto.\\
        
        \textbf{Palabras clave:} Blockchain, Alastria, Identidad Soberana, Ethereum, Quorum, Solidity, Smart Contracts, Hacking, Ciberseguridad, Seguridad de la Información, Auditoría\ldots

