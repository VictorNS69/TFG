In this section we will talk about future steps to be taken, either because they could not be done in this thesis or because they were out of the scope.\\

The first thing I would have liked to do is to audit the different main components mentioned in section \textit{3.4 Project Structure}. There are components that are very necessary to audit such as \textit{Alastria Node}, the \textit{Wallet} and the \textit{Entity} demo. Another point that should be studied is the use of other tools for analyzing and searching for vulnerabilities.\\

Regarding future steps in Alastria ID, some of them have already begun to be carried out, such as a new governance and role management, as well as a definition for the creation of the "first role". Some code quality improvements to the Alastria working team have also been mentioned, as well as a proposal to change the structure of the \textit{TypeScript} library. Another improvement that could be made to the Alastria ID model is the implementation of the library in different popular languages, such as \textit{Java} and \textit{Python}.\\

Finally, one of the ideas I had for the topic of this thesis is to make a malicious wallet. The idea of designing a malicious wallet, or falsifying a real wallet from the market, and studying its impact and how it can compromise the network (blockchain) is something I find very striking. Something that is also interesting and fun is to exploit the common vulnerabilities of Smart Contracts, such as gas theft or self-destruction attacks.\\

As we see, there are still many paths that can be studied and improved, both in the theorical model and in the implementation. Regarding audits, it is always a good idea to audit several times, with different tools and by different experts.
