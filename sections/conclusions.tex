 The Blockchain technology is a revolution because of the many advantages and features it brings with it.  This revolution consists of creating applications based on transparency, trust, immutability and security. On the other hand, new concepts emerge, such as that of \acrlong{ssi}. \acrlong{ssi} has a lot of potential, since it benefits both the user and the company; there are even benefits that have not been discovered yet.\\
    
    The problem with \acrlong{ssi}, as it is something new, implies that there are not many standards defined, so it is not easy to design a new model and even less to implement it. In fact, it is clear from the work done that, despite being the work of several excellent professionals, bugs and vulnerabilities still appear in implementations, and even more so in blockchain-related implementations, as this technology is also relatively new.\\
    
    With the study carried out, several models of \acrshort{ssi} have been found, each with its peculiarities and its strengths and weaknesses. \textit{David19} proposes citizen participation to address the current COVID-19 pandemic and future pandemics that may arise. \textit{Serto (uPort)} has implemented a wallet that allows users to manage their identity and exchange claims between participants. Alastria, on the other hand, has implemented an \acrlong{mvp} with the objective of showing the strengths of the \acrshort{ssi} and the Alastria ID model. Alastria has created a project with its own Smart Contracts, a library that facilitates the use of the Contracts for programmers who do not have extensive knowledge of blockchain technology, and demo implementations (not commercial) of a wallet and a entity website. Each project and model has peculiarities that others do not, so now, some associations like Alastria are seeking to improve interoperability with the other implementations and different blockchain networks, in order to create an ecosystem that does not depend on central entities.\\
    
    Concluding this work, I believe that Blockchain technology has much to contribute to the technological world and to actual society. Improvements in health, in online transactions, security, in traceability of goods, ... But as it is a new technology, it is possible that the projects are not safe and robust. That is why it is very important to audit all aspects of each blockchain-based implementation. From the typical audits of the machines that host the servers and nodes, to the \acrshort{api}s and front-ends, without forgetting the audits of the Smart Contracts and the code that is written.\\
    
    Finally, after having studied and understood in detail the \acrlong{ssi}, I think it is a concept that, already sounds strong, but will sound more in the coming years. It is a concept with enormous potential. Imagine to be able to control your identity and manage it, to be able to have control of who you share your data with, to be able to do onboarding more quickly and efficiently in different entities and web pages; it is something we really need and we don't know it yet.\\
    
    Personally, this thesis has meant a lot to me. I carried out this study while working for the first time. Along the way I have met many brilliant professionals, from different companies and with different profiles. I have improved as a developer, I have started to train as a cybersecurity professional, I have improved my skills in the technological and social area. To make this thesis I also had to learn and improve in many other points. I have learned new programming languages, new technologies; I have started to use the SCRUM methodology.\\
    
    In short, I have learned a lot, I have "fallen in love" with cybersecurity and I hope to continue fighting with blockchain and security.
